\documentclass{article}

\usepackage[cm]{fullpage}
\usepackage{hyperref}

\begin{document}

\title{90-minute Animint tutorial}
\author{Toby Dylan Hocking}
\maketitle
\thispagestyle{empty}

\section*{Animint book Ch2: basics of ggplot2}

\url{http://cbio.ensmp.fr/~thocking/animint-book/Ch02-ggplot2.html}

%\subsection*{Translating sketches to ggplots and animints}
\hrulefill

\textbf{Sketching} a data viz on paper can be useful, since the sketch
can be directly translated into R+ggplot2 code.

A \textbf{ggplot} consists of one or more geoms, each with its own
data set and aesthetic mapping.

An \textbf{animint} \texttt{viz} is a list of ggplots and options
which can be rendered via \verb|structure(viz, class="animint")|.

\textbf{Exercise:} re-make interactive scatterplot of World Bank data,
using a different aes mapping.

\hrulefill

%\subsection*{Multi-layer and multi-plot animints}

\textbf{Multi-layer} graphics consist of ggplots with several geoms.

A \textbf{multi-plot} animint \texttt{viz} list contains more than one
ggplot.

\textbf{Exercise:} create an animint with three ggplots.

\section*{Animint book Ch3: showSelected}

\url{http://cbio.ensmp.fr/~thocking/animint-book/Ch03-showSelected.html}

\hrulefill

\texttt{aes(showSelected=variable)} means to only show data for the
selected value of \texttt{variable}.

\textbf{Exercise:} add another geom or plot to the previous data viz.

\hrulefill

\textbf{Smooth transitions} can be specified using
e.g. \texttt{duration=list(year=2000)},\\and then for each geom with
\texttt{aes(showSelected=year)}, you should specify \texttt{aes(key)}.

\textbf{Animation} can be specified using
e.g. \texttt{time=list(variable="year", ms=2000)}.

\textbf{Exercise:} make a data viz with animation but without smooth
transitions.

\section*{Animint book Ch4: clickSelects}
\url{http://cbio.ensmp.fr/~thocking/animint-book/Ch04-clickSelects.html}

\hrulefill

\texttt{aes(clickSelects=variable)} means clicking the geom changes
the selected value of \texttt{variable}.

\textbf{Exercise:} make the country text label disappear when the
corresponding region is de-selected.

\hrulefill

\texttt{first=list(year=1970)} means that the year 1970 should be
selected when the data viz is first rendered.

\texttt{selector.types=list(country="multiple")} means to use multiple
selection for the \texttt{country} variable.

\textbf{Exercise:} make clicking a country text label de-select that
country.

\hrulefill

\verb|geom_tallrect| occupies the entire vertical space, so only
\texttt{aes(xmin,xmax)} need to be specified.

\textbf{Exercise:} add layers to the scatterplot, and animate the data viz.

\end{document}
