\documentclass[11pt]{article}
\usepackage[utf8]{inputenc}
\usepackage[T1]{fontenc}
\usepackage{fullpage}
\usepackage{graphicx}
\usepackage{grffile}
\usepackage{longtable}
\usepackage{wrapfig}
\usepackage{rotating}
\usepackage[normalem]{ulem}
\usepackage{amsmath}
\usepackage{textcomp}
\usepackage{amssymb}
\usepackage{capt-of}
\usepackage{hyperref}
\author{Toby Dylan Hocking}
\date{\today}

\begin{document}

\section*{``Understanding and creating interactive graphics,'' a tutorial for useR 2016}
%\tableofcontents

%% \section*{{\bfseries\sffamily TODO} }
%% \label{sec:orgheadline1}

%% send this to useR-2016@R-project.org before January 10, 2016.

\label{sec:orgheadline3}

\begin{center}
\begin{tabular}{llll}
Instructor & Institution & Address & Email\\
\hline
Toby Dylan Hocking & McGill Univ.  & Montreal, Canada & toby.hocking@mail.mcgill.ca\\
Claus Thorn Ekstrøm & Univ. Copenhagen & Copenhagen, Denmark & ekstrom@sund.ku.dk\\
\end{tabular}
\end{center}

\section*{Background}
\label{sec:orgheadline5}

An interactive graphic invites the viewer to become an active partner
in the analysis and allows for immediate feedback on how the data and
results may change when inputs are modified. Interactive graphics can
be extremely useful for exploratory data analysis, for teaching, and
for reporting.

Because there are so many different kinds of interactive graphics,
there has recently been an explosion in R packages that can produce
them. A beginner with little knowledge of interactive graphics can
thus be easily confused by (1) understanding what kinds of graphics
are useful for what kinds of data, and (2) finding an R package that
can produce the desired type of graphic. This tutorial solves these
two problems by (1) introducing a vocabulary of keywords for
understanding the different kinds of graphics, and (2) explaining what
R packages can be used for each kind of graphic.

\section*{Tutorial Overview}

Attendees will gain hands-on experience with using R to create
interactive graphics. We will discuss several example data sets and
several R interactive graphics packages. Attendees will learn a
vocabulary that helps to understand the strengths and weaknesses of
the many different packages which are currently available.

\section*{Detailed Outline}
\label{sec:orgheadline10}
\begin{itemize}
\item Differences between interactive/dynamic and non-interactive/static graphics.
\item A vocabulary for understanding interactive graphics in terms of
  complexity, possible actions, and effects.
\begin{description}
\item[Complexity] multi-panel, multi-layer, multi-plot.
\item[Actions] animation, direct manipulation (clicking, hovering), indirect manipulation
  (menus, buttons).
\item[Effects] zoom, highlight, show/hide (data, labels, tooltips),
  hyperlink.
\end{description}
\item High-level interactive plotting packages (rgl, MESS, rCharts,
  clickme, rMaps).
\item Discussion of frustrations that
new users unfamiliar with JavaScript may encounter when using these
libraries.
\item Interactive graphics with shiny, plotly, ggplot2.
\item Multi-layer graphics with ggplot2.
\item Multi-panel graphics with facets in ggplot2, useful in two
  different situations:
\begin{itemize}
\item same plot for different data subsets, and
\item different plots with aligned axes.
\end{itemize}
\item Animated graphics using the animation package.
\item Multi-panel, multi-layer, multi-plot, interactive graphics
  (animint, shiny, ggplot2, ggvis).
\end{itemize}

\section*{Background Knowledge}

Knowledge of base, lattice, or ggplot2
functions for creating static graphics will be helpful. No
knowledge of JavaScript is necessary.

\section*{Potential Attendees}
\label{sec:orgheadline12}
Our tutorial is appropriate for:
\begin{description}
\item[complete R newbies:] the vocabulary for understanding interactive
graphics should be useful, even if you don't understand all of the R
code that we present.
\item[intermediate useRs:] you already know how to use basic R data
  structures like data.frame and list, and maybe static graphics
  packages like ggplot2. You will learn how to create interactive
  graphics.
\item[interactive graphics package developeRs:] you have written an
  interactive graphics package. Please come to see us present some
  other packages, and to discuss the current state-of-the-art and
  future directions.
\end{description}

\section*{Requirements for interactive session}

Attendees should run the following R code to install the packages
which are required for our tutorial:

\begin{verbatim}
source("http://raw.githubusercontent.com/tdhock/interactive-tutorial/master/packages.R")
\end{verbatim}

\end{document}
