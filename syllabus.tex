\documentclass[11pt]{article}
\usepackage[utf8]{inputenc}
\usepackage[T1]{fontenc}
\usepackage{fullpage}
\usepackage{graphicx}
\usepackage{grffile}
\usepackage{longtable}
\usepackage{wrapfig}
\usepackage{rotating}
\usepackage[normalem]{ulem}
\usepackage{amsmath}
\usepackage{textcomp}
\usepackage{amssymb}
\usepackage{capt-of}
\usepackage{hyperref}
\author{Toby Dylan Hocking}
\date{\today}
\title{Understanding and producing interactive graphics}
\hypersetup{
 pdfauthor={Toby Dylan Hocking},
 pdftitle={},
 pdfkeywords={},
 pdfsubject={},
 pdfcreator={Emacs 24.3.1 (Org mode 8.3.2)}, 
 pdflang={English}}
\begin{document}

%\tableofcontents

%% \section{{\bfseries\sffamily TODO} }
%% \label{sec:orgheadline1}

%% send this to useR-2016@R-project.org before January 10, 2016.

\section{Tutorial Title}
\label{sec:orgheadline2}

Understanding and creating interactive graphics

\section{Instructors}
\label{sec:orgheadline3}

\begin{center}
\begin{tabular}{llll}
Name & Institution & Address & Email\\
\hline
Toby Dylan Hocking & McGill & Montreal, Canada & toby.hocking@mail.mcgill.ca\\
Claus Thorn Ekstrøm & Univ. Copenhagen & Copenhagen, Denmark & ekstrom@sund.ku.dk\\
\end{tabular}
\end{center}

\section{Brief Description of Tutorial}
\label{sec:orgheadline5}

An interactive graphic invites the viewer to become an active partner
in the analysis and allows for immediate feedback on how the data and
results may change when inputs are modified. Interactive graphics can
be extremely useful for exploratory data analysis, for teaching, and
for reporting.

Because there are so many different kinds of interactive graphics,
there has been an explosion in R packages that can produce them
(e.g. animint, shiny, rCharts, rMaps, ggvis, htmlwidgets). A beginner
with little knowledge of interactive graphics can thus be easily
confused by (1) understanding what kinds of graphics are useful for
what kinds of data, and (2) finding an R package that can produce the
desired type of graphic. This tutorial solves these two problems by
(1) introducing a vocabulary of keywords for understanding the
different kinds of graphics, and (2) explaining what R packages can be
used for each kind of graphic.

\section{Goals}
\label{sec:orgheadline6}

\begin{enumerate}
\item Explain and emphasize the role that interactive graphics have in
  exploratory data analysis, reporting, and teaching.
\item By showing several examples, explain different categories of
  graphics: animated, multi-panel, and multi-layer, interactive
  (direct vs indirect manipulation).
\item Explain how existing R packages can be used to create these
  different types of graphics.
\item Explain the strengths and weaknesses of the existing R packages, to
  highlight directions for future work.
\end{enumerate}

\section{Detailed Outline}
\label{sec:orgheadline10}

\subsection{A vocabulary for understanding interactive graphics, 30 minutes}
\label{sec:orgheadline7}

In this section we will give a high-level introduction about
interactive graphics, without going into details about R code for
specific packages.

Motivation: interactive graphics in exploratory data analysis,
reporting results and teaching.

Vocabulary for describing interactive graphics:
\begin{description}
\item[zoomable] reduce axes limits and hide data outside of limits. can
  do it with both static and interactive.
\item[{interactive}] user can change what is displayed without
  changing the code. (aka dynamic)
  \begin{description}
  \item[{direct manipulation}] interacting with plot elements (lines,
    points, etc).
  \item[{indirect manipulation}] interacting with keyboard, mouse clicks
    on widgets (buttons, menus, etc).
    Useful when there are many similar plots for
different data subsets,
    but you \textbf{don't} want to see them all at the same time.
\end{description}
\item[{animated}] An animated graphic automatically advances over time,
like a video. Animated graphics are most useful when data sets
have a time dimension. The only interaction possible is moving
forward and backward in time.
\item[{multi-layer}] A multi-layer graphic uses several geometric elements
to show several data sets and/or variables. Multi-layer plots are
useful for showing relationships between data sets and/or
variables.
\item[{multi-panel}] A multi-panel graphic shows different things in
different panels (sub-plots) which each have their own axes
(perhaps different from each other). Useful when there are many
similar plots for different data subsets, and you \textbf{do} want to
see them at the same time. Also useful for showing different
plots with aligned axes.
\end{description}
Compare and contrast:
\begin{description}
\item[{interactive vs animated}] only interaction possible in an animated
graphic is moving forward and backward in time (animated graphics
are thus a subset of interactive graphics).
\item[{interactive vs multi-panel}] both useful for many similar plots
with different data subsets. Do you want to see all the subsets
at the same time? (yes=multi-panel, no=interactive)
\end{description}

\subsection{Creating interactive graphics using R packages}
\label{sec:orgheadline9}

In this section we will show specific R code examples from the various
packages.

\begin{description}
\item[{High-level interactive plotting packages, 30 minutes}] \mbox{ }

\begin{itemize}
\item Simple approaches like rotating plots (rgl package) and simple user
interaction (wallyplot from MESS package).
\item Interactive bar plots (rCharts, several different JavaScript
interfaces, interfacing with JavaScript libraries to change axes
and legends)
\item Interactive scatter plots showing happiness and tax rate (rCharts,
and clickme packages, several different JavaScript interfaces, add
dropdown effects and improve tooltips)
\item interactive maps and choropleths (the rMaps packages)
\item Discussion of frustrations that new users unfamiliar with
JavaScript may encounter when interfacing with JavaScript libraries
\end{itemize}
\item[{Interactive graphics with shiny and plotly, 30 minutes}] \mbox{ }
\begin{itemize}
\item Teaching least squares estimation (shiny)
\item Teaching power calculations (shiny)
\item Reproducing some of the previous graphics on happiness and tax
rate in plotly (ggplot2, and ggplotly, adding tooltips/hover
effects, and dropdown)
\item Graphics on prediction accuracy for Danish population predictions
(plotly, adding sliders)
\end{itemize}
\item[{Multi-layer graphics, ggplot2 package, 15 minutes}] \mbox{ }
\begin{itemize}
\item A map that shows a circle for every city, and a line for borders of
each country.
\item A plot of a linear model that shows data as circles, a regression
line, and model residuals as line segments.
\end{itemize}
\item[{Multi-panel graphics, facets in ggplot2, 15 minutes}] useful in two
different situations:
\begin{description}
\item[{Same plot for different data subsets}] a linear model fit to each
of several data subsets.
\item[{Different plots with aligned axes}] World Bank data viz with one
time series panel, and one scatterplot panel.
\end{description}
\item[{Animated graphics, animation package, 15 minutes}] \mbox{ }
\begin{itemize}
\item Gradient descent (time=iterations).
\item Two-panel World Bank data viz (time=years).
\end{itemize}
\item[{Interactive + animated + multi-panel + multi-layer, 45 minutes}] a
few packages are able to produce complex graphics which can be
described by several vocabulary words.
\begin{description}
\item[{shiny + ggplot2}] World Bank data viz, interacting with widgets
changes selected year, countries, regions.
\item[{shiny + ggvis}] same kind of graphic with World Bank data.
\item[{animint}] World Bank data viz, direct manipulation changes
selected year, countries, regions.
\end{description}
\end{description}

\section{Background Knowledge}
\label{sec:orgheadline12}

Since we plan to present state-of-the-art interactive graphics, people
should know how to use R data structures (lists, data.frames) and the
ggplot2 package. 

Even though many examples will be interactive web graphics, we will
assume only knowledge of R, not HTML/JavaScript.

There are two classes of potential attendees:
\begin{itemize}
\item UseRs who are not very familiar with interactive graphics should
benefit the most, since we will give a high-level overview of many
different packages.
\item DevelopeRs of interactive graphics packages are encouraged to
  come, to discuss the current state-of-the-art and future directions.
\end{itemize}

\end{document}
